% Local IspellDict: en
\chapter{Template Manual}\label{ch02:manual}

In the following, you will find some general information on how to use this template as well as some tips on how to apply some features of \LaTeX\ to write not only a conclusive but also well-structured thesis. This is not meant as a general manual on how to learn \LaTeX, but to remind you of certain features and to guide you.

One general hint: try to avoid a situation in which only 1 or 1 1/2 lines of a paragraph are printed on the next page. These snippets can easily be missed by your reader and it is no nice way of typesetting.

Sometimes, words which are not known by \LaTeX\ are separated wrongly or are written on the margins of the page. Use the \combrac{hyphenation}{com-mand} in the preambel of the \texttt{main.tex}-file to show \LaTeX\ how words can be separated properly.

\section{Text Structure}\label{ch02:sec1:structure}

When structuring an academic work in particular the criterion \emph{completeness} has to be observed. This, however, should be achieved without any redundancies and overlap. The entire work should follow a common theme and each section, even each argument, should logically follow this theme. Furthermore, the theme has to be evident to readers, who should be able to know at any time whilst reading at which point of the argument they are. In order to achieve this, it may be handy to develop a continuous example throughout the writing.

In this thesis, the usual structures like \com{chapter}, \com{section}, \com{subsection} etc.\ can be used. Depending on the length of your thesis, it is not recommended to use more than these three subdivisions. In order to refer to the chapters, use the \combrac{label}{label} and related \combrac{ref}{label}-commands, e.g., write \verb|see Chapter~\ref{ch02:manual}| but not \verb|see \ref{ch02:manual}|.  (The same holds for tables, figures, etc.: State to what type of thing you refer \emph{and} capitalize that thing \emph{and} use the tilde character to create a non-breaking space before the number.)

In \LaTeX, several commands like \combrac{textit}{text} for italic,  \combrac{textbf}{text} for bold text, \combrac{textsc}{text} for small caps, \combrac{underline}{text} for underlined text, or even \combrac{texttt}{text} for typewriter text can be used to highlight information. Please use only a limited number of highlights and stick to one type of highlight for a specific purpose!

Texts can further be structured by using lists. The common list-environments \texttt{itemize}, \texttt{enumerate} and \texttt{description} also work in this template.

% Another way to structure information is to use footnotes for including further information. The common command \combrac{footnote}{note} should be placed without space right after the last word of the sentence before the punctuation\footnote{E.g., like this.}.
% The previous is not correct: https://english.stackexchange.com/questions/9632/footnote-marks-at-end-of-a-sentence

\section{Colours}\label{ch02:sec2:color}

This template comes with the colors defined in the Corporate Design of the University of Münster, ERCIS, and the Research Group CSSSA. Table~\ref{ch02:tab:colors} lists the color names and shows the resulting color. You can apply them to text by using the \combrac{textcolor\{color name\}}{text} command. If you want to use them in other images or graphics, you will also find the hexadecimal codes below. Do not forget to add an \# before you use them in other applications.

\begin{table}[ht]
    \centering
    \begin{tabular}{@{}lcc@{\hskip 1cm}lcc@{}}
        \toprule
        {\bf Color Name} & {\bf Result} & {\bf Hex Code} & {\bf Color Name} & {\bf Result} & {\bf Hex Code} \\ \toprule
            wi-green1 & \textcolor{wi-green1}{\blacksmiley{}} & 5bbb56 &
            wi-mud & \textcolor{wi-mud}{\blacksmiley{}} & b5bb56 \\
            wi-iceblue & \textcolor{wi-iceblue}{\blacksmiley{}} & 56bbb7 &
            wi-grey & \textcolor{wi-grey}{\blacksmiley{}} & 929292 \\         wi-darkblue & \textcolor{wi-darkblue}{\blacksmiley{}} & 567cbb &
            wwu-black     & \textcolor{wwu-black}{\blacksmiley{}} & 3e3e3b \\
            wi-purple    & \textcolor{wi-purple}{\blacksmiley{}} & 880085 &
            wwu-lightgreen  & \textcolor{wwu-lightgreen}{\blacksmiley{}} & 7ab516 \\
            wi-pink  & \textcolor{wi-pink}{\blacksmiley{}} &  f36196 &
            wwu-green      & \textcolor{wwu-green}{\blacksmiley{}} & 008e96 \\
            wi-coral     & \textcolor{wi-coral}{\blacksmiley{}} & f88379 &
            wwu-lightblue      & \textcolor{wwu-lightblue}{\blacksmiley{}} & 009dd1 \\
            wi-rose & \textcolor{wi-rose}{\blacksmiley{}} & ffc0cb &
            wwu-blue     & \textcolor{wwu-blue}{\blacksmiley{}} & 006e89 \\
            wi-ocher & \textcolor{wi-ocher}{\blacksmiley{}} & bbb056 &
            ercis-red      & \textcolor{ercis-red}{\blacksmiley{}} & 852339 \\
			\bottomrule
    \end{tabular}
    \caption{Representation of color codes defined in this template.}\label{ch02:tab:colors}
\end{table}

\section{Floats}\label{ch02:sec3:floats}

In order to include float objects, the usual commands for inserting a   \texttt{table}, \texttt{figure} or \texttt{listing} / \texttt{algorithm} can be used. For figures, tables as well as code, proper captions, and labels should be set. Similar to chapters and sections, labels can be set to refer to the floating object. Captions should help the reader what kind of information is displayed as well as why is it displayed. This template does not include a list of figures or a list of tables. Again, please keep in mind that these missing lists are missing on purpose and should only be included in this template if your supervisor asks for it.

Additionally, you can determine where images and tables should be placed on the page. In scientific publications, it is the state-of-the-art to place them on top of the page. If it fits in your case, you should also stick to this rule of thumb. Use the square brackets after the definition of the environment to force \LaTeX{} to place the floats where you want them to be: \texttt{h} for here, \texttt{t} for at the top of the page and \texttt{b} for the bottom of the page. You can also combine them in order to give \LaTeX{} some margin for maneuver.

Tables need nested environments. First, a \combrac{begin}{table} environment is used, in which a \combrac{begin}{tabular} environment is called. The \texttt{tabular} environment is used to define the table itself. Please make sure that all information is structured properly so that the reader can directly see and understand the displayed information. A guide for really nice tables can be found in the documentation of the \texttt{booktabs} package which is also included in this template \citep{Fear2020booktabs}.

\begin{table}[ht]
    \centering
    \begin{tabular}{@{}llr@{}}
        \toprule \multicolumn{2}{c}{Article} \\
        \cmidrule(r){1-2} Number & Item & Price (\euro)\\
        \midrule
        1 & Cheese & 3.50 \\
        3 & Steaks & 25.00 \\
        2 & Mangos & 4.00 \\ \bottomrule
    \end{tabular}
    \caption{An exemplary table using the rules defined in the \texttt{booktabs} package by \citeauthor{Fear2020booktabs}.}\label{tab:my_label}
\end{table}

% Merged with the DBIS template (Section 2.5.2)
For included images, make sure to use high-resolution images (vector) images or that you generate images by using the \texttt{TikZ/PGF} package in \LaTeX, the \texttt{ggplot} package in \texttt{R} or the \texttt{plotly} package in \texttt{python}. They can be included by placing them in a \verb|\begin{figure}|-environment and by using the \combrac{includegraphics[width]}{file} command, which can handle many formats. PDF files are preferable for vector graphics.  PNG files are the best way of including pixel graphics, such as graphs or diagrams not available in vector format.  JPEG files may be used to include photographic or artistic images.

\begin{figure}[t]
    \centering
    \includegraphics[width=0.6\textwidth]{figures/dummy_figure.png}
    \caption{Always choose a proper label and caption which describes what the reader can see. You should write complete sentences and not only keywords. This nice image was generated by using the \texttt{ggplot}-package of \texttt{R}.}\label{fig:my_label}
\end{figure}

The last type of float which can be included are code snippets. You can either include pseudocode to outline the mechanisms of algorithms or real source code.

Pseudocode should be created by using the \texttt{algorithmicx} package. It provides some commands to structure the code and is easy to handle. As tables, the outer environment is the \texttt{algorithm} environment, while the inner environment is an \texttt{algorithmic}-environment. Further information on how to use the commands can be found in the documentation of the \texttt{algorithmicx} package \citep{Janos2020algorithmicx}.

\begin{algorithm}
    \begin{algorithmic}[1]
        \Procedure{Euclid}{$a,b$} \Comment{required b > a}
        \State $r\gets a\bmod b$
        \While{$r\not=0$}
              \State $a\gets b$
              \State $b\gets r$
              \State $r\gets a\bmod b$
           \EndWhile\label{euclidendwhile}
           \State \textbf{return} $b$
        \EndProcedure
    \end{algorithmic}
    \caption{Euclid’s algorithm in pseudo-code}\label{ch02:code:pseudo-euclid}
\end{algorithm}

For source code, the package \texttt{minted} is already included in this thesis. It provides features to highlight keywords in code for several programming languages. Simply  include code files via the \combrac{inputminted\{language\}}{file} command in a \texttt{listings} or \texttt{algorithm} environment. If it is just one single line, use \combrac{mint}{language}\texttt{|Line of Code|}.

Be aware that if you want to use the \texttt{minted} package on your local computer and not in Overleaf, you might have to install some packages on your local machine. For further information, take a look at the package documentation \citep{Poore2020minted}.\label{ch02:sec3:minted}

% \begin{algorithm}[ht]
%     \inputminted[
%     xleftmargin=20pt,
%     framesep=2mm,
%     baselinestretch=1.2,
%     linenos
%     ]{python}{code/euclid.py}
%     \caption{Euclid’s algorithm in python-code}\label{ch02:code:python-euclid}
% \end{algorithm}

\section{Subfloats}

Sometimes it makes sense to place figures side by side. Use the \texttt{subfloat} environment inside a \texttt{figure} environment if you want to reference the images separately. If not, you can also use two commands without an additional environment by including two \combrac{includegraphics[width]}{image} commands and adjust the width so that the images are placed side by side. Note that the sizes of either the subfigures or the images should be together lower than 1, e.g., both \texttt{width=0.49\com{textwidth}}.

\begin{figure}[ht]
    \centering
    \begin{subfigure}{0.49\textwidth}
        \centering
        \includegraphics[width=0.8\textwidth]{figures/dummy-subfigure1.png}
        \caption{The caption of the first image.}\label{ch02:fig:fig1a}
    \end{subfigure}
    \begin{subfigure}{0.49\textwidth}
        \centering
        \includegraphics[width=0.8\textwidth]{figures/dummy-subfigure2.png}
        \caption{The caption of the second image.}\label{ch02:fig:fig1b}
    \end{subfigure}\label{ch02:fig:fig1}
    \caption{Two interesting figures side by side which can be referenced independently. These images were generated by using the \texttt{plotly} extension for \texttt{R}.}
\end{figure}

\section{Bibliography and Citations}

Bibliographies are an important part of any piece of academic writing. On the one hand they enable readers to easily extend research on the given work.  On the other hand they help writers to correctly address their sources as well as keeping track of them.  With \LaTeX{}, bibliographic information is managed in \texttt{.bib} files (e.g., see \texttt{library.bib} coming with this template).

\LaTeX{} provides several commands to reference sources, which are then included in the bibliography at the end of the document.  This template uses \href{http://ctan.org/pkg/biblatex}{biblatex} (with \texttt{biber} as backend) and the option \texttt{natbib}.  The style of references and bibliography depends on your setting of \texttt{chair} in the beginning of the main document.  Please make sure to set that option first.  Then, different variants of citation commands may look as follows (others exist as well):

Use authors as subject in proper English sentence. \combrac{textcite}{Goodfellow2016deep} define \ldots produces: \textcite{Goodfellow2016deep} define \ldots

\begin{itemize}
\item Use authors as subject in proper English sentence. For instance, using the command  \combrac{textcite}{Goodfellow2016deep} define \ldots you can produce: \textcite{Goodfellow2016deep} define \ldots
\item Note how more than three authors are automatically replaced with “et al.”. \combrac{textcite}{Goodfellow2014gan} define \ldots produces: \textcite{Goodfellow2014gan} define \ldots
\item Use reference at end of sentence or as object:
  \begin{itemize}
  \item \ldots~\combrac{citep}{Goodfellow2014gan}. produces: \ldots~\citep{Goodfellow2014gan}.
  \item as stated in~\combrac{citep}{Goodfellow2014gan}, \ldots produces: as stated in~\citep{Goodfellow2014gan}, \ldots
  \end{itemize}
\item Include page numbers, e.g., ~\combrac{citep[p.~2]}{Codd1970relational} produces: \citep[p.~2]{Codd1970relational}
\item Produce just author names, e.g., ~\combrac{citeauthor}{Goodfellow2016deep} produces: \citeauthor{Goodfellow2016deep}
\end{itemize}

Generally, each source has to be named. How this can be done varies between institutions and departments.
Relevant literature can be included in the \texttt{library.bib} file in the main folder of this template. Inside, you will also find some examples on which information is needed for which type of literature.

If authors or book titles are mentioned by name, the source should be cited right after them, e.g.: The book \emph{Artificial Intelligence: A Modern Approach}~\citep{Rusell2003artificial} has been written by S. J.\ Russel and P.\ Norving.

Note that sources should \emph{not} just be listed at the end of a paragraph.  Instead, sources should be cited when they are used. For instance, the following sentence might be the first one in a paragraph on Generative Adversarial Networks: According to~\citep{Goodfellow2014gan}, Generative Adversarial Networks share the following major advantages (followed by text based on that source).

\section{Quotes}
The following example shows two things at once.  Firstly, how to
change the language environment, secondly, how to quote directly.

Direct quotes are produced by using the \combrac{begin}{quote} environment, as shown in the example below.

\begin{quote}
    Future users of large data banks must be protected from having to know how the data is organized in the machine (the internal representation).~\citep[p.~2]{Codd1970relational}
\end{quote}

Of course, copied text should be kept at a minimum and needs to be cited properly.  Next to the \verb|quote| environment just shown, quotation marks can be used: ``Future users of large data banks\ldots''~\citep[p.~2]{Codd1970relational}.
Importantly, \LaTeX{} knows language dependent opening and closing quotation marks.  In source files, you can either use correct UTF-8 symbols (if your editor supports this) or use \verb|``| and \verb|''| for opening and closing quotation marks in English (or \verb|"`| and \verb|"'| in German).

Quotes in a different language are set using the \combrac{begin}{otherlanguage} environment, as illustrated below.

\begin{quote} % TODO: change the quote to one related to Machine Learning
  \begin{otherlanguage}{ngerman}
    Aus logischer Sicht besitzt ein Von-Neumann-Rechner neben der CPU
    einen \emph{Speicher}, welcher begrifflich zusammengesetzt ist aus
    einem ROM- und einem RAM-Teil.~\citep[p.~230]{OberschelpVossen}
  \end{otherlanguage}
\end{quote}

\section{Licensing}\label{sub:licensing}
When you create an own work, reuse of your work is restricted by
copyright laws.  E.g., if you write a good or excellent thesis, not
even your supervisors are allowed to create copies of your work for
classroom use.  To enable reuse of (academic and other) works,
Creative Commons (CC) provides a set of licenses.  For Open
Educational Resources (OER), the license variants CC BY-SA, CC BY,
and CC0 are particularly useful.  Please consult CC’s web page
\url{https://creativecommons.org/licenses/} to learn more about
different license variants.

The \texttt{license} option in the introductory \texttt{documentclass}
command allows you to publish your work under such a free and open
license.  This will create a short license statement towards the end
of your thesis (on the page with your declaration of authorship):
“\cLizenzTemplate{Your-Chosen-Variant}”

Recall that you must always cite your sources.  This is particularly
relevant if you want to publish your work under a CC license and you
include resources that you did not create yourself (e.g., figures):
You \emph{must} indicate (and, of course, respect) license terms for
included resources so that others know which license terms apply
where.

\section{Headings}
As mentioned above, if a section is broken further down, at least two subheadings shall follow.

\subsection{Subheadings}
This means a structure like this is \emph{NOT} appreciated for this is the only subheading. Therefore, there is no need to have it in the first place.

\section[Blacklist]{Blacklist of commands and packages}
You can include packages in the preamble of the \texttt{main.tex}-file if you need more than already defined in this template. But below you will also find a list of packages which should not be included in this document since, e.g., some of them interfere with existing packages. Also, all packages and commands which alter the design of the thesis, margins, indents or any other spaces are especially forbidden.

Disallowed packages:
\begin{description}
    \item[\combrac{usepackage}{authblk}] This package changes the default settings of the title page.
    \item[\combrac{usepackage}{subfig}] This package interferes with the already loaded \texttt{float} package.
    \item[\combrac{usepackage}{fullpage}] This package interferes with the already loaded \texttt{geometry} package.
    \item[\combrac{usepackage}{grffile}] Use proper file names, then you do not need this package.
    \item[\combrac{usepackage}{wrapfig}] Text is not allowed to float around figures.
\end{description}

Disallowed commands:
\begin{description}
    \item[\textbackslash{}addtolength] You are not allowed to change any margins or other page styles.
    \item[\textbackslash{}baselinestretch] You are not allowed to change any margins or other page styles.
    \item[\textbackslash{}tiny] This is not an acceptable font size.
    \item[\textbackslash{}vspace] No negative value may be used in proximity of a caption, figure, table, section, subsection, subsubsection, or reference.
    \item[\textbackslash{}vskip] No negative value may be used to alter spacing above or below a caption, figure, table, section, subsection, subsubsection, or reference.
\end{description}

%%% Local Variables:
%%% mode: latex
%%% TeX-master: "../main_thesis"
%%% TeX-command-extra-options: "-shell-escape"
%%% End:
