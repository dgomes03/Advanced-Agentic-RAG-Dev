% Local IspellDict: en
\chapter{Brief Guide to Academic Practice and Writing}\label{ch03:academicguide}
At the University of Münster, we strive to follow rules for Good Scientific
Practice\footnote{\url{https://www.uni-muenster.de/forschung/en/foerderung/kodex.shtml}}.
Please familiarize yourself with the General Principles summarized in
§1.  Although these principles are phrased for “academic staff”, they
apply to your work as student as well.

For Bachelor students, we recommend that you take part in our
“Vorbereitungskurs Bachlorarbeit”, which follows a book written at
our department \citep{Bergener2019wissenschaftliches}.  In addition,
Alfred Brink teaches academic writing at the University of Münster (and has written a book
on that topic \citep{Brink2013anfertigung}).

\section{Parts of an Academic Writing}
\begin{description}
	\item[Abstract] The abstract is a short summary of the work at hand.  Its purpose is to give a concise overview of the work's contents and enable potential readers to judge the relevance of the work.  To achieve this, abstracts of academic writings should state the purpose and aims of the work (including some context information, if necessary), the methods applied, and the main findings in approximately 100--150 words.  By default, this template does not produce an abstract.  Add one if your supervisor asks you to do so.

	\item[Introduction] The introduction should clearly motivate and outline the aims of the work and its relevant adjoining subject areas. The most important terms have to be defined exactly, which requires a precise wording. Furthermore, the research question and research method can be briefly hinted at. By highlighting the process of the study conducted the structure of the work can be outlined and justified. In many case it has proven to be sensible having a figure at the end of the introduction showing the structure, line of argument, and key message of the work. This section does not necessarily have to be called introduction but may be titled in regard to the work.

	For theses, an introduction should be one to three pages. If basic definitions take up too much space, it may be opted to add another section in between the introduction and the main body (e.g., fundamentals) which elaborates on basic principles. For other types of writing, the introduction should be shorter according to its overall page count.

	From experience, it can be stated that it makes sense to write the introduction last. It may be drafted at the beginning but during the course of writing it is likely that changes to the original plan occur (the sequence and relation of sections may change or new aspects appear). By writing the introduction last, unnecessary work is avoided.

	\item[Main body] The main body should be split into several chapters or sections and is concerned with the actual topic of the work. Typical sections (amongst others) are: literature review, methodology, and findings.

	\item[Conclusion] To conclude the work a summary of the main findings should be given. Additionally, it should be elaborated on the limitations of the work at hand and how future research may address these as well as other open issues. Furthermore, possible future developments can be pointed out.
\end{description}

\section{Good Practices}

For any scientific work, you need to understand on whose shoulders you
might stand.  Thus, first, carefully read the basic literature.  Then,
clarify comprehension question (such as technical terms and
abbreviations).  It is also important to understand the context of
articles and of the underlying theory.  For example, an article
regarding cryptography can be either in the context of mathematical
application or in that of Internet security.  To analyze a certain
paper, it is important to identify the core ideas of the authors: What
do they highlight as being new discoveries?  What are the most
important points?  What is good or bad about their work?  Once this
has been done, their central theme and the differences between their
work and the work of others should be highlighted.  Also, the work
should be critically reflected: Has the article, for instance, been
written for a particular vendor?  Are there opposing opinions?  Is the
content applicable in the way the authors suggest?

Importantly, correct spelling and grammar should be applied.  When
writing in German, frequent mistakes involve missing commas, for which
rules~\citep{kommaregeln} exist, and
Deppenleerzeichen~\citep{deppenleerzeichen}.  When writing in English,
American or British rules must be used consistently.

Also, correct citation as shown in this template has to be applied,
while you need to make sure that \BibTeX{} entries contain necessary
fields.  Every reference (literature or Web) that is named in the
according list of references has to be mentioned in the text but no
particular ration of Web sources to text book or journal sources is
required.  Please note that in this guide we did not reference each
work in the lists of references; rather, the aim was to give an
example what the lists should look like.

%%% Local Variables:
%%% mode: latex
%%% TeX-master: "../main_thesis"
%%% TeX-command-extra-options: "-shell-escape"
%%% End:
